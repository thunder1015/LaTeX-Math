\documentclass[11pt,oneside,a4paper]{article}
% 导言区配置
\usepackage{ctex}
\usepackage{amsmath}
\usepackage{amssymb}
\usepackage{amsthm}
\usepackage{graphicx}
% \usepackage{hyperref}
% \usepackage{geometry}
% 生成封面
\title{矩阵及其运算}
\author{朝暮}
\date{\today}

\begin{document}
\maketitle
\newtheorem{definition}{定义}
\newtheorem{theorem}{定理}
\setlength{\abovedisplayskip}{1pt}
\setlength{\belowdisplayskip}{5pt}

\begin{theorem}[带 Lagrange 余项的 Taylor 多项式]
设函数 $f(x)$ 在点 $x_0$ 的某邻域 $O(x_0)$ 内具有 $n+1$ 阶导数,则对任意 $x \in O(x_0)$,存在 $\xi$ 介于 $x$ 与 $x_0$ 之间,使得

\begin{equation*}
    f(x) = f(x_0) + \frac{f'(x_0)}{1!}(x - x_0) + \frac{f''(x_0)}{2!}(x - x_0)^2 + \cdots + \frac{f^{(n)}(x_0)}{n!}(x - x_0)^n + r_n(x).
\end{equation*}
其中,$r_n(x) = \frac{f^{(n+1)}(\xi)}{(n+1)!}(x - x_0)^{n+1}$ 称为 Lagrange 余项。
\end{theorem}

\begin{proof}[证明]$\qquad$
    \begin{enumerate}
        \item 先介绍一种构造性的证明。见菲赫金哥尔茨《微积分学教程》、陈纪修《数学分析》等。
        
        固定 $x_0 \in O(x_0)$,构造辅助函数

        \begin{equation*}
            \begin{aligned}
                F(t) &= f(x) - \left( f(t) + f^{\prime}(t) (x-t) +\frac{f^{\prime}(t)}{2!}(x-t)^2 +\cdots+\frac{f^{(n)(t)}}{n!}(x-t)^{n}\right) \\
                &=f(x) - \sum_{k=0}^{n} \frac{f^{(k)}(t)}{k!}(x-t)^k.
            \end{aligned}
        \end{equation*}

        \begin{equation*}
            H(t) = (x-t)^{n+1}.
        \end{equation*}
    \end{enumerate}
\end{proof}
\end{document}