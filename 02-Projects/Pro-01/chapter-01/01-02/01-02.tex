\documentclass[11pt,oneside,a4paper]{article}
% 导言区配置
\usepackage{ctex}
\usepackage{amsmath}
\usepackage{amsthm}
\usepackage{graphicx}
% \usepackage{hyperref}
% \usepackage{geometry}
% 生成封面
\title{引言:高等代数的研究对象}
\author{朝暮}
\date{\today}

% 文档正文
\begin{document}
\maketitle
\newtheorem{definition}{定义}
\newtheorem{example}{例}


作为高等代数的第一节课,需要先搞清楚一件事情——我们可以从这门课程中学到什么。换言之,高等代数在研究什么?

我们在中学阶段就已经开始接触``代数''了,对``代数''的第一印象就是所谓的``用字母表示数''。

实际问题中常含有一些``已知量''与``未知量''。

\begin{itemize}
    \item[*] \textbf{未知量}:通常用 $x,y,z,\cdots$ 等字母来表示;
    \item[*] \textbf{已知量}:如果是具体的问题,已知量通常是具体的数字;如果是抽象问题,已知量一般用 $a,b,c,\cdots$ 来表示。 
\end{itemize}

基于实际问题,可以找出等量关系,进而列出含有未知量的等式,这就是\textbf{方程}。

\paragraph{一元一次方程} 在初中一年级,我们首先接触到的就是一元一次方程,即形如

\begin{equation*}
    ax+b =c
\end{equation*}
的方程。

\begin{proof}
    当 $a\ne 0$ 时,可以简单地通过四则运算,得到方程的解:

    \begin{equation*}
        x = \frac{c-b}{a}.
    \end{equation*}
\end{proof}

值得注意的是,上述一元一次方程的解是一个``未知量在左边''的等式,这被称为一元一次方程的\textbf{求根公式}。

\paragraph{一元二次方程} 在初中三年级,我们又接触了一元二次方程,即形如

\begin{equation*}
    ax^2 + bx + c = 0
\end{equation*}
的方程。

\begin{proof}
    显然,一元二次方程也存在求根公式。

    \begin{itemize}
        \item 若$a > 0$ 且判别式 $\Delta = b^2-4ac\ge 0$,方程有两个解,可以用如下的求根公式来表示:
        \begin{equation*}
            x = \frac{-b \pm \sqrt{b^2 - 4ac}}{2a}.
        \end{equation*}
        \item 若 $a \ne 0$ 且判别式 $\Delta = b^2-4ac =0$,方程有唯一解。
        \item 若 $a = 0$,方程\textbf{退化} 为一元一次方程。
    \end{itemize}
\end{proof}



\paragraph{一元高次方程} 我们将方程中非零项的最高次数称为该方程的\textbf{次数},抓住方程``次数''的特征,自然而然地可以将其推广至一元三次方程、一元四次方程等。方便起见,我们将次数高于 $2$ 的方程统称为\textbf{一元高次方程}。

一元高次方程是否有根?这是经典代数学的中心问题。具体来说,需要关心这样一些事情:

\begin{itemize}
    \item 一元高次方程是否存在根?
    \item 如果存在根,是否可求?如果可求,是否存在统一的求根公式?
\end{itemize}


前面的内容是通过抓住方程``次数''的特征,从一次方程推广到高次方程,进而引出经典代数学的中心问题。然而实际问题往往复杂得多,比如方程所涉及到的未知量常常不止一个,下面我们就抓住方程未知量的个数,也就是``元''的特征进行推广。


\paragraph{二元一次方程}含有两个未知量,且每个未知量都是一次的,称这样方程为\textbf{二元一次方程}。如

\begin{equation*}
    ax+by+c = 0.
\end{equation*}

\paragraph{三元一次方程}含有三个未知量,且每个未知量都是一次的,称这样的方程为\textbf{三元一次方程}。如

\begin{equation*}
    ax+by+cd + e = 0.
\end{equation*}

\paragraph{一元$n$次方程} 一般地,若方程含有$n$个未知量,且每个未知量都是一次的,则称这样的方程为\textbf{一元 $n$ 次方程}。


可以考虑一元$n$次方程的几何意义。联想中学阶段学习过的平面解析几何知识:一个二元一次方程可以对应到平面直角坐标系中的一条直线,

\begin{example}
    考虑一元二次方程 $3x+2y=1$ 在平面直角坐标系中的图像。

    \centering
    \includegraphics[scale=0.9]{figures/figure01.png}
\end{example}

可以借用几何直观,通常可将$n$元一次方程称作\textbf{$n$元线性方程}。

显然,方程的未知量个数变多后,仅凭一个方程是无法求解出来未知量的,需要多个方程,进而组成了所谓的``方程组''。

\paragraph{二元一次方程组} 有两个一元二次方程构成的方程组。如 

\begin{equation*}
    \begin{cases}
        a_{11}x_1 + a_{12}x_2 = b_1 \\
        a_{21}x_1 + a_{22}x_2 = b_2
    \end{cases}
\end{equation*}

\paragraph{三元一次方程组} 有三个一元二次方程构成的方程组。如 

\begin{equation*}
    \begin{cases}
        a_{11}x_1 + a_{12}x_2 + a_{13}x_3 = b_1 \\
        a_{21}x_1 + a_{22}x_2 + a_{23}x_3 = b_2 \\
        a_{31}x_1 + a_{32}x_2 + a_{33}x_3 = b_3
    \end{cases}
\end{equation*}

以上是我们在中学阶段所熟知的关于一元二次、三次方程组的形式。需要特别指出的是,方程组中所含方程的个数与方程未知量的个数并无必要联系。如三元一次方程组,可以包含三个方程、也可以包含四个,或者更多。

\begin{example}
    \begin{equation*}
        \begin{cases}
            x_1+3x_2+x_3=2\\
            3x_1+4x_2+2x_3=9\\
            -x_1-5x_2+4x_3=10\\
            2x_1+7x_2+x_3=1\\
        \end{cases}
    \end{equation*} \label{exam-2}
\end{example}


自然而然地,可以抓住方程组``元''的特征,将其推广到任意多个未知量的情形。

\paragraph{n元一次方程组} 一般地,由若干个一元$n$次方程构成的方程组称为$n$元一次方程组。借用几何直观,可将其称作$n$元线性方程组。其一般形式如下:

\begin{equation}
    \begin{cases}
        a_{11}x_1+a_{12}x_2+\cdots +a_{1n}x_n=b_1\\
        a_{21}x_1+a_{22}x_2+\cdots +a_{2n}x_n=b_2\\
        \cdots \cdots\\
        a_{s1}x_1+a_{s2}x_2+\cdots +a_{sn}x_n=b_s\\
    \end{cases}\label{n-bodys}
\end{equation}

其中,$a_{ij}$($1\le i \le s,1\le j \le n$)称为方程组(\ref{n-bodys})的\textbf{系数};$b_{j}$($1\le j \le n$)称为其\textbf{常数项}。

如何对$n$元线性方程组进行求解呢?在中学阶段,我们已经学习过如何对二元、三元线性方程组进行求解,常使用的方法是:代入消元法、加减消元法。可以肯定的是,这两种方法对于一般的$n$元线性方程组依然有效。为了使其更加规范、乃至可以使用现代计算机进行求解,后续将引入所谓的``矩阵消元法''。

\paragraph{注} 计算机的发展,也极大地推动了线性方程组理论的发展。许多过去难解的、乃至解不了的方程组,如今凭借计算机可以方便得进行求解。


回顾对二元、三元线性方程组的求解过程,有以下的共同点:

\begin{enumerate}
    \item 仅仅对系数项以及常数项进行了运算;
    \item 由字母表示的未知量并未参与运算。
\end{enumerate}

为了简化书写,可以将方程组的所有系数``抽离''出来。

\begin{example}
    将 例\ref{exam-2}方程组的所有系数抽离出来,按照原有次序,可以排列成如下的数表。

    \begin{equation*}
        \left[ \begin{matrix}
            1 & 3 & 1 & 2 \\
            3 & 4 & 2 & 9 \\
            -1 & -5 & 4 & 10 \\
            2 & 7 & 1 & 1
        \end{matrix}\right]
    \end{equation*}
\end{example}


对于一般的$n$元线性方程组,可以按照同样的操作,将其所有的系数项按原有次序排列成一张数表,这就是``矩阵''。


\begin{equation*}
    \left\{ \begin{array}{c}
        a_{11}x_1+a_{12}x_2+\cdots +a_{1n}x_n=b_1\\
        a_{21}x_1+a_{22}x_2+\cdots +a_{2n}x_n=b_2\\
        \cdots\cdots\\
        a_{s1}x_1+a_{s2}x_2+\cdots +a_{sn}x_n=b_s\\
    \end{array} \right. \xrightarrow{\text{抽离系数}}\left[ \begin{matrix}
        a_{11}&		a_{12}&		\cdots&		a_{1n}\\
        a_{21}&		a_{21}&		\cdots&		a_{2n}\\
        \vdots&		\vdots&		\ddots&		\vdots\\
        a_{s1}&		a_{s2}&		\cdots&		a_{sn}\\
    \end{matrix} \right] 
\end{equation*}

\paragraph{矩阵} 由$s\times n$个数排列成的$s$行、$n$列的数表称为矩阵,通常用大写字母$A$、$B$、$C$、$\cdots$来表示一个矩阵,如

\begin{equation}
    A = \left[\begin{matrix}
        a_{11}&		a_{12}&		\cdots&		a_{1n}\\
        a_{21}&		a_{21}&		\cdots&		a_{2n}\\
        \vdots&		\vdots&		\ddots&		\vdots\\
        a_{s1}&		a_{s2}&		\cdots&		a_{sn}\\
    \end{matrix}\right] 
\end{equation}

\begin{example} 称
    \begin{equation*}
        A = \left[\begin{matrix}
            a_{11}&		a_{12}&		\cdots&		a_{1n}\\
            a_{21}&		a_{21}&		\cdots&		a_{2n}\\
            \vdots&		\vdots&		\ddots&		\vdots\\
            a_{s1}&		a_{s2}&		\cdots&		a_{sn}\\
        \end{matrix}\right] 
    \end{equation*}
    为方程组(\ref{n-bodys})的\textbf{系数矩阵}。
\end{example}

\begin{example}称
    \begin{equation}
        \bar A = \left[\begin{matrix}
            a_{11}&		a_{12}&		\cdots&		a_{1n}&     b_1\\
            a_{21}&		a_{21}&		\cdots&		a_{2n}&     b_2\\
            \vdots&		\vdots&		\ddots&		\vdots&     \vdots\\
            a_{s1}&		a_{s2}&		\cdots&		a_{sn}&     b_s\\
        \end{matrix}\right] 
    \end{equation}
    为方程组(\ref{n-bodys})的\textbf{增广矩阵}。
\end{example}

高等代数的出发点之一,就是$n$元线性方程组的求解问题。而研究$n$元线性方程组的解法,可以归结为对矩阵的研究。


然而,仅仅了解了一般的$n$元线性方程组如何求解是远远不够的。面对一个具体的线性方程组,我们期望可以在求解之前,就能预先对其解的情况做一定的判断:是否有解?若有解,有多少解?特别是无解的方程组,解了半天才发现解不存在,就太浪费时间了。

如何在实际求解前,就能得知方程组解的情况呢?可以借助于几何直观。

\begin{example}
    在平面几何中,二元一次方程可以表示一条直线,由两个二元一次方程构成的方程组就对应两条直线。由几何关系可知:平面内的两条直线可以平行;可以相交;可以重合。如果直线相交,则交点正是该方程组的解。
    
    \centering
    \includegraphics[scale=0.75]{figures/figure02.png}
\end{example}

借助于几何直观,可以猜测,对于一般的$n$元线性方程组,其至少存在三种解的情况:有唯一解;有无穷多解;无解。原则上来讲,是否存在其它解的情况是需要进一步分析讨论的。比如恰好有两个解;恰好有三个解等等。

如何建立这种方程组解的情况判别方法呢?要想找到一种统一的、普遍的判别方法,就不能拘泥于某些特例,而要从所有方程组的共通点着手。显而易见,方程组的系数项以及常数项可以作为突破点。而出于简化的目的,结合前面的讨论,我们的任务又可归结为研究方程组的系数矩阵以及增广矩阵。

方程组的系数矩阵以及增广矩阵均由若干行构成,每一行又由若干个、按照一定次序排列的数构成。这里出现了``若干个数字的有序排列'',而类似的结构并不陌生。例如:平面坐标$(x,y)$;空间坐标$(x,y,z)$;物理时空坐标$(z,y,z,t)$……

综合这些示例,可以抽象出``$n$元有序数组''的概念。


\paragraph{n维向量空间} 考虑由所有$n$元有序组构成的集合,并且类比几何空间,可在其上引入加法、数量乘法两种运算,进而得到\textbf{$n$维向量空间}。

进一步的分析表明,研究线性方程组解的判别方法,可以促使我们研究$n$维向量空间,这将在第3章展开细致讨论。

此外,若$n$元线性方程组有无穷多解,则还可以通过研究$n$维向量空间的结构,来研究解集的结构。

对于几何空间、$n$维向量空间,仍可抓住共同特征,抽象出``线性空间''的概念。

既然已经通过$n$维向量空间解决了线性方程组的求解问题,为什么还要继续抽象出所谓的``线性空间''的概念呢?这就是数学的一大特点,当解决了某一类问题之后,应当沿着某一思路继续探索,以期得到新的理论,使其能够处理更广泛的问题。

\end{document}

